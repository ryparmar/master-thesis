\begin{introduction}
\label{section:introduction}
    The Italian philosopher Gianni Vattimo, who combines the perspectives of a philosopher and a sociologist in his work, came up with the concept of a \emph{trasnparent society}.~\parencite{vattimo_2013} The concept can be understood as another name for contemporary society, which is also referred to as society of mass communication. According to Vattimo, one of the factors that made the transition from modern to so-called postmodern society possible was the emergence of mass media, which continue to play a decisive role in shaping it. 
    
    Mass media was expected to make the world clearer, more \enquote{transparent} and create a more enlightened society. Nevertheless, it is they that, according to Vattimo, characterize this society as more chaotic and complex. With the vast amount of information that society generates, which can spread very quickly and efficiently thanks to mass media and social networks, it is impossible to achieve anything like a fully informed view of the world. The existence of that volume of information and data also allows the media to choose what information to report and how to frame it. In this way, as Vattimo writes, they will not only present a certain image of the world to their consumers, but also create it. However, is the image of the world thus constructed necessarily the real one, or at least sufficiently representative?
    
    Such a phenomenon can be illustrated on the Czech media scene during the war in Syria and the subsequent migration wave, which was a major source of struggle for the entire European Union at the time. In 2018, the Czech media published over 80,000 articles and reports on refugees and migration.~\parencite{prokop_2020} This was roughly one article for every two refugees and migrants arriving in Europe via the Mediterranean. By comparison, there were only about 20,000 articles and reports on the climate crisis (including articles on the melting of glaciers, carbon dioxide emissions, etc.) or on foreclosures and debt collection, which began to emerge as serious problems for Czech society in this period.~\parencite{prokop_2020}

    Fake news also contributes significantly to the aforementioned opacity, chaos and information overload. Fake or otherwise misleading news may be created intentionally or unintentionally, may be part of an organised disinformation campaign or may be mere solitary acts. In any case, this is one of the problems that is increasingly heard in the public space. In the context of fake news, there are very often calls for better education, more media lessons or work with critical thinking. These are all very important elements that should have a place in the educational process or at least be discussed. However, in the era of deep-fakes and the rapid improvement of machine-generated texts~\parencite{hao_2020}, it is also necessary to use these technologies and more advanced tools, which are now often mentioned in connection with the notion of automatic fact-checking.
    
    The existence of a huge amount of information and data and the possibility of their very effective distribution to the desired target group are strong preconditions for the effective dissemination of disinformation. At the same time, these are precisely the same prerequisites that make it possible to combat it effectively.
    
    Automated fact-checking cannot be thought of as a silver bullet to solve the problems of misinformation, populism and information overload. But rather as a tool to help people get their bearings and save valuable mental capacity. In that sense, it can also be any tool that helps journalists, media analysts and other professionals process data more efficiently, making their work better and more effective, which hopefully helps us all. The goal of this paper is to contribute to the development of such a fact-checking tool, specifically the part of the tool that is tasked with finding relevant documents from a large collection for a given query.
    
    \epigraphfontsize{\small\itshape}
    \epigraph{"The Democrats don't matter. The real opposition is the media. And the way to deal with them is to flood the zone with shit."~\parencite{illing_2020}}{--- \textup{Steve Bannon}, Advisor to Donald Trump in January-August 2017}
    
    % \noindent
    % The thesis is organized into following chapters:
    % \begin{itemize}
    %     \item Fact-checking -- introduces definition of the fact-checking task, goal of this work and related work;
    %     \item Background -- provides a definition of document retrieval (DR) task and description of DR methods;
    %     \item Datasets -- describes used datasets;
    %     \item Proposed Solution -- provides and overview of our proposed solution and models we experimented with; 
    %     \item Experiments -- presents the experimental setup of models and a comparison of results of tested DR methods;
    %     \item Conclusion -- summarizes the work and suggests directions for future work.
    % \end{itemize}

%https://www.technologyreview.com/2021/03/11/1020600/facebook-responsible-ai-misinformation/?truid=2e4bfd4345985354cbe21af25f5160cc&utm_source=the_download&utm_medium=email&utm_campaign=the_download.unpaid.engagement&utm_term=&utm_content=03-11-2021&mc_cid=a5caea9a4f
\end{introduction}